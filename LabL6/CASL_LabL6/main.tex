%%%%%%%%%%%%%%%%%%%%%%%%%%%%%%%%%%%%%%%%%%%%%%%%%%%%%%%%%%%%%%%%%%%%%%%
% Based on IEEE the conference template available                     %
% at https://www.ieee.org/conferences/publishing/templates.html       %
% Adapted for the Data Science Lab course at Politecnico di Torino    %
% by Giuseppe Attanasio, Flavio Giobergia                             %
% 2020, DataBase and Data Mining Group                                %
%%%%%%%%%%%%%%%%%%%%%%%%%%%%%%%%%%%%%%%%%%%%%%%%%%%%%%%%%%%%%%%%%%%%%%%

\documentclass[conference]{IEEEtran}
\usepackage{cite}
\usepackage{amsmath,amssymb,amsfonts}
\usepackage{algorithm}
\usepackage{algorithmic}
\usepackage{graphicx}
\usepackage{textcomp}
\usepackage{xcolor}
\usepackage{subfigure}

\begin{document}

\title{
Lab L6: Dynamic processes on graphs
}

\author{
    \IEEEauthorblockN{Emanuele Pietropaolo}
    \IEEEauthorblockA{
        \textit{Politecnico di Torino} \\
        Student id: s319501 \\
        emanuele.pietropaolo@studenti.polito.it
        }
}

\maketitle
\begin{abstract}
A dynamical process is a mathematical model that describes the evolution of a system.
%
When run on a graph, it can model complex real-world scenarios such as pandemics or voter preferences.
%
In this context, the nodes represent the state of the system and the edges represent the possible transitions that the system can make. 
%
Simulating how such a system evolves over time can provide useful insights into how the modelled system behaves, more than analytical analyses can. 

This paper presents different simulations of the \textit{Voter Model} on random graphs (\textit{Erdős-Rényi model}) and on $Z^2$ and $Z^3$ graphs to see how these types of graphs behave in different conditions, how the initial state and the size influence the time to reach consensus and the +1 consensus probability.

\end{abstract}

\section{Problem overview}

%Random graphs are a special type of graph that can be described by a probability distribution, or random process, that generates them. 
%
%Their number of edges depends on a stochastic process starting from a given number of nodes. 
%
%The process involves connecting random nodes with an edge based on probability. 
%
%The most important and widely used model describing this type of graph is called the \textbf{Erdős-Rényi model}.

\section{Proposed approach}

    \subsection{Algorithm to generate samples of G(n,p) graphs}

    This report examines two methods of generating samples of random graphs:
    \begin{itemize}
        \item \textbf{Generating directly an Erdős–Rényi graph}: this method involves iterating over all the nodes and for each one iterating over all the nodes not already seen. This method then create an edge between a pair of nodes based on the \textit{edge probability}. This algorithm has complexity $O(n^2)$ but it tends to $O(nlog(n))$ because the second for loop is decreasing over time.
        \begin{algorithm}
            \caption{Erdős–Rényi method}
            \label{alg:first}
            \begin{algorithmic}
                \STATE \textbf{Input:} n = number of nodes, p = edge probability
                \FOR{$i$ \textbf{in} $\text{range}(n)$}
                    \FOR{$j$ \textbf{in} $\text{range}(i + 1, n)$}
                        \IF{$\text{random}() < p$}
                            \STATE create edge(i,j)
                        \ENDIF
                    \ENDFOR
                \ENDFOR
            \end{algorithmic}
        \end{algorithm}

        \item \textbf{Take advantage in case of small edge probability}: when p can be considered small in front of the number of nodes, it can be calculated the number of expected edges (\textit{m}) that the graph will have through the formula:
        \begin{center}
            \begin{math}
                \frac{n*(n-1)*p}{2}
            \end{math}            
        \end{center}
        
        Then create a for loop that for each edge picks two random nodes and connects them. This algorithm has complexity $O(m)$ that for small values of $p$ tends to $O(n)$.
        \begin{algorithm}
            \caption{Taking advantage of small edge probability}
            \label{alg:second}
            \begin{algorithmic}
                \STATE \textbf{Input:} n = number of nodes, p = edge probability
                \STATE $m = n*(n-1)*p / 2$
                \FOR{$i$ \textbf{in} $\text{range}(m)$}
                    \WHILE{True}
                        \STATE i,j = generate two nodes index
                        \IF {$i==j$ \OR $j in node[i][neighbors]$}
                            \STATE \textbf{continue}
                        \ELSE
                            \STATE $create edge(i,j)$
                            \STATE \textbf{break}
                        \ENDIF
                    \ENDWHILE
                \ENDFOR
            \end{algorithmic}
        \end{algorithm}
    \end{itemize}

    \subsection{Data structure}

    I used a \textbf{dictionary} to handle the graph. Each key of the dict is an index of a node and each node is a nested dictionary with a list of neighbors and a state variable.

    \subsection{How the FES is handled}

    The FES handle the wake-up events and at the beginning is filled up with only one event, the wake-up for a random node. 
    %
    Then after each wake-up event, it's filled with a new of such events associated with a new random node.
    %
    The wake-up process is supposed to follow a Poisson process so the inter-times are generated from a exponential distribution with $\lambda=1$

    I modelled the \textbf{Voter Model} so for each event is updated the state variable of the waked up node ($v$) with the value of a random neighbor ($w$), following the formula:
    \begin{center}
        \begin{math}
            x_v(t_v^+) = x_w(t_v^-)
        \end{math}
    \end{center}

\section{Results}


%\begin{figure}[h!]
%    \centering
%    \includegraphics[width=6.5cm]{media/state_2(biased).png}
%    \label{fig:state_2(biased)}
%\end{figure}

%In Fig. \ref{fig:state_2} and \ref{fig:state_2(biased)} we can see that in the second case, it is more likely that the graph will reach a stable state starting from a biased condition. 
    
\section{Conclusion}

In conclusion, the proposed approach demonstrated that can successfully generate random graphs.

%\bibliography{bibliography}
%\bibliographystyle{ieeetr}

\end{document}
