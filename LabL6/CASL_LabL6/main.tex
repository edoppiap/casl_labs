%%%%%%%%%%%%%%%%%%%%%%%%%%%%%%%%%%%%%%%%%%%%%%%%%%%%%%%%%%%%%%%%%%%%%%%
% Based on IEEE the conference template available                     %
% at https://www.ieee.org/conferences/publishing/templates.html       %
% Adapted for the Data Science Lab course at Politecnico di Torino    %
% by Giuseppe Attanasio, Flavio Giobergia                             %
% 2020, DataBase and Data Mining Group                                %
%%%%%%%%%%%%%%%%%%%%%%%%%%%%%%%%%%%%%%%%%%%%%%%%%%%%%%%%%%%%%%%%%%%%%%%

\documentclass[conference]{IEEEtran}
\usepackage{cite}
\usepackage{amsmath,amssymb,amsfonts}
\usepackage{algorithm}
\usepackage{algorithmic}
\usepackage{graphicx}
\usepackage{textcomp}
\usepackage{xcolor}
\usepackage{subfigure}

\begin{document}

\title{
Lab L6: Dynamic processes on graphs
}

\author{
    \IEEEauthorblockN{Emanuele Pietropaolo}
    \IEEEauthorblockA{
        \textit{Politecnico di Torino} \\
        Student id: s319501 \\
        emanuele.pietropaolo@studenti.polito.it
        }
}

\maketitle
\begin{abstract}
A dynamical process is a mathematical model that describes the evolution of a system.
%
When run on a graph, it can model complex real-world scenarios such as pandemics or voter preferences.
%
In this context, the nodes represent the state of the system and the edges represent the possible transitions that the system can make. 
%
Simulating how such a system evolves over time can provide useful insights into how the modelled system behaves, more than analytical analyses can. 

This paper presents different simulations of the \textit{Voter Model} on random graphs (\textit{Erdős-Rényi model}) and on $Z^2$ and $Z^3$ graphs to see how these types of graphs behave in different conditions, how the initial state and the size influence the time to reach consensus and the +1 consensus probability.

\end{abstract}

\section{Problem overview}

The Voter Model is a mathematical model that describes a dynamic process.
%
It models the evolution of opinions in a group of people.  
%
As we simulate the Voter Model on to a graph, each node of the graph represents an individual and the edges between nodes represent the connection that each person has with the others.

This model suits the situation where there are two different opinions and individuals can be influenced by what their connections think about them. 
%
The opinion that a person has is represented by a state variable that can take two values: $ x_v(t) \in \{-1,1\}$.
%
The change in opinion is modelled as an event where a node wakes up and changes its state variable by copying the state variable of one of its neighbours, following the formula:

\begin{center}
    \begin{math}
        x_v(t_v^+) = x_w(t_v^-)
    \end{math}
\end{center}

\section{Proposed approach}

Simulating the voter model over a graph involves several stages. 
%
First you need to generate the graph you want to use, and then you start modelling the events that make the system evolve.

    \subsection{Data structure for the graph}

    A \textbf{dictionary} was used to implement the graph. 
    %
    Each node is accessible by an index and contains a nested disctionary where the state variable and the list of neighbours are stored. 

    Each state variable is initialised according to the probability of being in the $+1$ state, this probability was called the \textit{bias probability}. 

    \subsection{Algorithm to generate samples of $G(n,p)$ graphs}
    
    One of the most common types of graph is the random graph. 
    %
    It is described by a probability distribution or random process that generates it. 
    %
    In fact, its number of edges depends on a stochastic process that connects the nodes based on a probability, the \textit{edge probability}. 
    %
    The model that almost exclusively describes a random graph is the Erdős Rényi (ER) model. 

    When the ER graph is generated, the dictionary is initialised with the desired number of nodes, without first connecting them. 
    %
    Then, to create the connection according to the edge probability, one of two methods are used based on the difference (in terms of factor of 10) between the number of nodes and the edge probability:
    \begin{itemize}
        \item \textbf{Bernoulli experiment for each pair of nodes}: this method involves iterating over all the nodes, and for each one iterates over all the nodes not yet seen. 
        %
        This method then creates an edge between a pair of vertices based on the \textit{bias probability}.
        %
        This algorithm has a complexity of $O(n^2)$, but tends to be $O(nlog(n))$ because the second for loop is decreasing in time.

        \item \textbf{Take advantage of low edge probability}: if p can be considered small before the number of nodes, the number of expected edges ($m$) that the graph will have can be calculated by the formula:
        \begin{center}
            \begin{math}
                  (n*(n-1)*p)/2 
            \end{math}            
        \end{center}        
        where $p$ is the \textit{edge probability}. 
        %
        Then, for each edge, two random nodes that are not already connected are selected and connected.
        %
        This algorithm has complexity $O(m)$, which tends to $O(n)$ for small values of $p$.
    \end{itemize}

    \subsection{Algorithm to generate samples of $Z^2$ and $Z^3$ graphs}

    A graph can also be a \textit{Z-dimensional lattice}, where each node has an edge with all the nearest nodes in space. 
    %
    In this paper we will explore the $Z^2$ and $Z^3$ grids. The nodes of these graphs are arranged in such a way so that they form a 2D and 3D grid respectively, and where a node has all edges perpendicular and parallel to the others.


    \subsection{How the FES is handled}

    The FES handle the wake-up events and at the beginning is filled up with only one event, the wake-up for a random node. 
    %
    Then after each wake-up event, it's filled with a new of such events associated with a new random node.
    %
    The wake-up process is supposed to follow a Poisson process so the inter-times are generated from a exponential distribution with $\lambda=1$

\section{Results}


%\begin{figure}[h!]
%    \centering
%    \includegraphics[width=6.5cm]{media/state_2(biased).png}
%    \label{fig:state_2(biased)}
%\end{figure}

%In Fig. \ref{fig:state_2} and \ref{fig:state_2(biased)} we can see that in the second case, it is more likely that the graph will reach a stable state starting from a biased condition. 
    
\section{Conclusion}

In conclusion, the proposed approach demonstrated that can successfully generate random graphs.

%\bibliography{bibliography}
%\bibliographystyle{ieeetr}

\end{document}
